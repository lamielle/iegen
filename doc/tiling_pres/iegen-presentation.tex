\documentclass[t,handout]{beamer}
%\documentclass[t,handout]{beamer} %use this header to print slides
\setbeamertemplate{navigation symbols}{}
\usepackage{beamerthemeshadow}

\usepackage{listings}
\lstset{
language=C,                % choose the language of the code
basicstyle=\LARGE,    % the size of the fonts that are used for the code
showstringspaces=false,    % underline spaces within strings
numbers=right,             % where to put the line-numbers
numberstyle=\normalsize,   % the size of the fonts that are used for the
                           % line-numbers
stepnumber=1,              % the step between two line-numbers. If it's 1 each
                           % line will be numbered
numbersep=-8pt,            % how far the line-numbers are from the code
%backgroundcolor=\color{white},  % choose the background color
showspaces=false,          % show spaces within strings adding particular
                           % underscores
showtabs=false,            % show tabs within strings adding particular
tabsize=3,                 % tab width
                           % underscores
%escapeinside={\%*}{*)}    % if you want to add a comment within your code
captionpos=b               %caption's placed at the bottom of the listing ('b')
}


\title[IEGen: Automatically Generating Inspectors and Executors]{IEGen}
\subtitle{Automatically Generating Inspectors and Executors}
\author{Alan LaMielle}
\institute{Colorado State University \\
Tiling Meeting}
\date{November 5, 2008}

%Show the table of contents after each section
%\AtBeginSubsection[]
%{
%	\begin{frame}<beamer>
%		\frametitle{Outline}
%		\tableofcontents[currentsection,currentsubsection]
%	\end{frame}
%}

\begin{document}

\begin{frame}
\titlepage
\end{frame}

\begin{frame}
\frametitle{Overview}
\tableofcontents
\end{frame}

\section{The Polyhedral Framework}

\subsection{Representation}
\begin{frame}
\frametitle{An example polyhedral computation}
\includegraphics<1->[scale=.6]{poly_comp}
\end{frame}
\begin{frame}[fragile]
\begin{exampleblock}{An example polyhedral computation}
\texttt{\lstinputlisting{poly_comp.c}}
\end{exampleblock}
\end{frame}
\begin{frame}[fragile]
\begin{exampleblock}{An example polyhedral computation}
\texttt{\lstinputlisting[basicstyle=\large]{poly_comp.alpha}}
\end{exampleblock}
\end{frame}

\begin{frame}
Iteration space:
\bigskip{}
\begin{itemize}
\item Constraint Representation:
$\left[\begin{array}{c}
1\\
-1\end{array}\right]\left[\begin{array}{c}
i\end{array}\right]\ge\left[\begin{array}{c}
1\\
n-1\end{array}\right]$
\bigskip{}
\item Vertex/Ray Representation: $(1),(n-1)$
\bigskip{}
\item Set/Relation Syntax: $\left\{ \left[i\right]:1\le i\le n-1\right\} $
\end{itemize}
\end{frame}

\begin{frame}
Data spaces of \texttt{x} and \texttt{fx}:
\bigskip{}
\begin{itemize}
\item Constraint Representation:
$\left[\begin{array}{c}
1\\
-1\end{array}\right]\left[\begin{array}{c}
i\end{array}\right]\ge\left[\begin{array}{c}
0\\
n-1\end{array}\right]$
\bigskip{}
\item Vertex/Ray Representation: $(0),(n-1)$
\bigskip{}
\item Set/Relation Syntax: $\left\{ \left[i\right]:0\le i\le n-1\right\} $
\end{itemize}
\end{frame}

\begin{frame}
Accesses: \\
\bigskip{}
\texttt{S1}'s first access of \texttt{x}:
\bigskip{}
$\left[\begin{array}{cc}
1 & 0\\
1 & 0\\
0 & 1\\
0 & 1\\
1 & -1\\
-1 & 1\end{array}\right]\left[\begin{array}{c}
i\\
i'\end{array}\right]\ge\left[\begin{array}{c}
1\\
n-1\\
1\\
n-1\\
-1\\
1\end{array}\right]$ \\
\bigskip{}
Other accesses are similar...
\end{frame}

\subsection{Transformation}
\begin{frame}
\begin{itemize}
\item Unimodular Transformation Framework: \\
\bigskip{}
$\left[-1\right]\left[i\right]=\left[-i\right]$
\bigskip{}
\item Kelly-Pugh Transformation Framework: \\
\bigskip{}
$\left\{ \left[i\right]\rightarrow\left[i'\right]:i'=n-1-i\right\}$
\end{itemize}

\end{frame}

\subsection{Generation}
\begin{frame}
\LARGE
Code Generators:
\begin{itemize}
\item CLooG
\item Omega
\item TLOG
\item HiTLoG
\end{itemize}
\bigskip{}
Common libraries:
\begin{itemize}
\item Polylib
\item PIP
\end{itemize}
\end{frame}

\subsection{Scope}
\begin{frame}
\begin{center}\begin{tabular}{|c|c|}
\hline
Dwarf & Support\tabularnewline
\hline
\hline
Dense Linear Algebra & \checkmark \tabularnewline
\hline
Sparse Linear Algebra & \ding{55} \tabularnewline
\hline
Spectral Methods & \checkmark \tabularnewline
\hline
N-Body Methods & \ding{55} \tabularnewline
\hline
Structured Grids & \checkmark \tabularnewline
\hline
Unstructured Grids & \ding{55} \tabularnewline
\hline
Monte Carlo & \checkmark \tabularnewline
\hline
Combinational Logic & \ding{55} \tabularnewline
\hline
Graph Traversal & \ding{55} \tabularnewline
\hline
Dynamic Programming & \checkmark \tabularnewline
\hline
Backtrack and Branch \& Bound& \ding{55} \tabularnewline
\hline
Construct Graphical Methods & \ding{55} \tabularnewline
\hline
Finite State Machines & \ding{55} \tabularnewline
\hline
\end{tabular}
\end{center}
\end{frame}

\section{The Sparse Polyhedral Framework}

\subsection{Representation}
\begin{frame}
\frametitle{An example sparse computation}
\center{\includegraphics<1->[scale=.5]{sparse_comp}}
\end{frame}
\begin{frame}[fragile]
\begin{exampleblock}{An example sparse computation}
\texttt{\lstinputlisting{sparse_comp.c}}
\end{exampleblock}
\end{frame}

\begin{frame}
\LARGE
The following needs to be specified:
\begin{itemize}
\item Symbolic Constants
\item Data Spaces
\item Computation inputs/outputs
\item Statements with iteration spaces and scattering functions
\item Access Relations
\item Data dependences
\item \emph{Index Arrays (Uninterpreted Functions)}
\end{itemize}
\end{frame}

\subsection{Transformation}
\begin{frame}
Three types acting on data spaces and iteration spaces:
\bigskip{}
\begin{tabular}{|c||c|c|}
\hline 
 & Data Space & Iteration Space\tabularnewline
\hline 
Permutation & CPack & LexMin\tabularnewline
\hline 
Projection & ? & ?\tabularnewline
\hline 
Embedding & Smashing & Tiling\tabularnewline
\hline
\end{tabular}\\
\bigskip{}
Specify the relation from the original data/iteration space to the new data/iteration space.
\end{frame}

\subsection{Generation}
\begin{frame}
Omega?

Other reordering specific Inspector/Executor generators.
\end{frame}

\subsection{Scope}
\begin{frame}
\begin{center}\begin{tabular}{|c|c|}
\hline
Dwarf & Support\tabularnewline
\hline
\hline
Dense Linear Algebra & \checkmark \tabularnewline
\hline
Sparse Linear Algebra & \checkmark \tabularnewline
\hline
Spectral Methods & \checkmark \tabularnewline
\hline
N-Body Methods & \checkmark \tabularnewline
\hline
Structured Grids & \checkmark \tabularnewline
\hline
Unstructured Grids & \checkmark \tabularnewline
\hline
Monte Carlo & \checkmark \tabularnewline
\hline
Combinational Logic & \ding{55} \tabularnewline
\hline
Graph Traversal & \ding{55} \tabularnewline
\hline
Dynamic Programming & \checkmark \tabularnewline
\hline
Backtrack and Branch \& Bound& \ding{55} \tabularnewline
\hline
Construct Graphical Methods & \ding{55} \tabularnewline
\hline
Finite State Machines & \ding{55} \tabularnewline
\hline
\end{tabular}
\end{center}
\end{frame}

\subsection{Why SPF?}
\begin{frame}
Supports far more application domains

Still relies on similar theory

Can utilize existing tools for code generation
\end{frame}

\section{IEGen}
\begin{frame}
Demo time!
\end{frame}

\begin{frame}
How do fuzzy dependences in Alphabets compare with UFSs in the SPF?
\end{frame}

%\subsection{Representation}
%\begin{frame}
%\end{frame}

%\subsection{Transformation}
%\begin{frame}
%\end{frame}

%\subsection{Generation}
%\begin{frame}
%\end{frame}

\end{document}
